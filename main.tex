\documentclass{apiDoc}

\title{API Pix}
\author{INSITUIÇÃO DE PAGAMENTO}
\date{\today}

% Documento principal
\begin{document}

\makecover

% Título
\title{Documentação da API Pix 2.0.0}
\author{Banco Central do Brasil}
\date{Outubro de 2020}
\maketitle

\tableofcontents

\section{Introdução}
O Pix é uma solução de pagamento instantâneo desenvolvida pelo Banco Central do Brasil. Este documento descreve os detalhes técnicos para integração com a API Pix 2.0.0.

\section{Credenciais e Autenticação}
\subsection{Geração de Credenciais}
Para acessar a API, as credenciais OAuth2.0 devem ser geradas. Inclua informações como Nome, CNPJ e certificado digital reconhecido.

\subsection{Token de Acesso}
\endpoint{POST}{/auth/server/oauth/token}{Obtém um token de acesso}{grant\_type=client\_credentials}

\section{Endpoints}
\subsection{Gerar Cobrança}
\endpoint{PUT}{/cob/{txid}}{Cria um QR Code dinâmico para cobrança.} {"calendario": {"expiracao": "3600"}, "valor": {"original": "123.45"}}

\begin{lstlisting}
{"calendario": {"expiracao": "3600"}, "valor": {"original": "123.45"}}
\end{lstlisting}


\subsection{Consultar Cobrança}
\endpoint{GET}{/cob/{txid}}{Consulta informações de uma cobrança existente.}{Nenhum parâmetro no corpo da requisição.}

\section{Exemplos de Código}
\subsection{Exemplo de Requisição}
\begin{lstlisting}
POST /auth/server/oauth/token HTTP/1.1
Host: qrpix-h.bradesco.com.br
Content-Type: application/x-www-form-urlencoded
Authorization: Basic {Base64(client_id:client_secret)}

grant_type=client_credentials
\end{lstlisting}

\section{Códigos de Resposta}
\begin{tabular}{|c|l|}
\hline
\textbf{Código} & \textbf{Descrição} \\
\hline
200 & Sucesso \\
400 & Erro de validação \\
401 & Não autorizado \\
500 & Erro interno do servidor \\
\hline
\end{tabular}

\section{Glossário}
\begin{itemize}
  \item \textbf{TxID} - Identificador único para a cobrança.
  \item \textbf{QRCode} - Código gráfico para pagamentos Pix.
\end{itemize}

\end{document}
